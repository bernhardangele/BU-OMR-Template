\documentclass[
  % Babel language, also used to load translations
  english,
  % Use A4 paper size, you can change this to eg. letterpaper if you need
  % the letter format. The normal methods to modify the paper size should
  % be picked up by SDAPS automatically.
  % a4paper, % setting this might break the example scan unfortunately
  % letterpaper
  %
  % If you need it, you can add a custom barcode at the center
  %globalid=SDAPS,
  %
  % And the following adds a per sheet barcode at the bottom left
  %print_questionnaire_id,
  %
  % You can choose between twoside and oneside. twoside is the default, and
  % requires the document to be printed and scanned in duplex mode.
  %oneside,
  %
  % With SDAPS 1.1.6 and newer you can choose the mode used when recognizing
  % checkboxes. valid modes are "checkcorrect" (default), "check" and
  % "fill".
  %checkmode=checkcorrect,
  pagemark,
  stamp]{sdapsclassic}
\usepackage[utf8]{inputenc}
% For demonstration purposes
\usepackage{multicol}
\usepackage{forloop}
\newcounter{question_number}

\author{Bournemouth University, Department of Psychology}
\title{Optical Mark Recognition (OMR) form}

\begin{document}
  % Everything you do should be done inside the questionnaire environment.

  % If you don't like the default text at the beginning of each questionnaire
  % you can remove it with the optional [noinfo] parameter for the environment 
  \begin{questionnaire}[noinfo]
    % There is a predefined "info" style to hilight some text.
    \begin{info}
      Please see instructions on the exam sheet.
    \end{info}

    % Use \addinfo to add metadata (which is printed on the report later on)
    \addinfo{Date}{22.11.2017}

    \section{Student and Exam information}
  \begin{multicols}{3}
      \textbox{1.2cm}{Name}
      \textbox{1.2cm}{Exam}
      \textbox{1.2cm}{Date}
  \end{multicols}
\begin{multicols}{2}
    \begin{choicegroup}{Please write the seven digits of your i- or s-Number (without the initial letter) into the $\sqcup$s and check the corresponding box below each one. Only check one box in each column.\columnbreak}
	
      % We have to add the possible choices at the start.
       \groupaddchoice{\Huge$\sqcup$}
       \groupaddchoice{\Huge$\sqcup$}
       \groupaddchoice{\Huge$\sqcup$}
       \groupaddchoice{\Huge$\sqcup$}
       \groupaddchoice{\Huge$\sqcup$}
       \groupaddchoice{\Huge$\sqcup$}
       \groupaddchoice{\Huge$\sqcup$}
       \groupaddchoice{\Huge$\sqcup$}

      % After that it is possible to add each question.
       \choiceline{\hfill 1}
       \choiceline{\hfill 2}
       \choiceline{\hfill 3}
       \choiceline{\hfill 4}
       \choiceline{\hfill 5}
       \choiceline{\hfill 6}
       \choiceline{\hfill 7}
       \choiceline{\hfill 8}
       \choiceline{\hfill 9}
       \choiceline{\hfill 0}
    \end{choicegroup}
\end{multicols}
    \section{Questions}
\begin{multicols}{3}

    \begin{choicegroup}{}
      % We have to add the possible choices at the start.
      \groupaddchoice{A}
      \groupaddchoice{B}
      \groupaddchoice{C}
      \groupaddchoice{D}

      % After that it is possible to add each question.
      \forloop[1]{question_number}{1}{\value{question_number} < 81}{%
      \choiceline{Q\arabic{question_number}}
}
    \end{choicegroup}

    \end{multicols}

  \end{questionnaire}
\end{document}

